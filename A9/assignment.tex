\documentclass[11pt]{report}
\usepackage{graphicx}
\usepackage{tabularx}
\usepackage{graphicx}
\usepackage[margin=0.5in]{geometry}
\usepackage{listings,xcolor}
\usepackage{tcolorbox}
\usepackage{multicol}
\tcbuselibrary{listings,skins}
\lstset{
    string=[s]{"}{"},
    stringstyle=\color{blue},
    comment=[l]{:},
    commentstyle=\color{black},
    breaklines=true,
}
\lstdefinestyle{mystyle}{
numbers=left, 
numberstyle=\small, 
numbersep=4pt, 
language=Python
}
\newtcblisting{mylisting}[2][]{
    arc=0pt, outer arc=0pt,
    listing only, 
    listing style=mystyle,
    title=#2,
    #1
    }
\PassOptionsToPackage{hyphens}{url}\usepackage{hyperref}
\begin{document}

\title{Assignment 7}
\author{Joshua Graham}

\maketitle
\pagebreak
\begin{abstract}
All Scripts used in the assignment can be found in the A9 folder, if needed.

\end{abstract}
\section{Problem 1 Process}
	I had a lot of trouble with this assignment. The first hard part I tackled was figuring out how I was going to link name and index. I cannot remember where I saw it, it was probably online but I cannot find the reference I used. I figured out I could do it with:
\begin{mylisting}[hbox,enhanced,drop shadow]{Linking by index}
#print(blogs.index("F-Measure"))
\end{mylisting}

The second tackle I had was fixing numpredict to stop throwing errors on strings. After reading the error dialog, I found I just had to get rid of the section argument in the list inside of numpredict.py. Below is the changed function:\textbf{I believe all that changed was vec2=data[i]}. This previously had a ["title"] argument beside it.
\begin{mylisting}[hbox,enhanced,drop shadow]{Linking by index}
def getdistances(data,vec1):
  distancelist=[]

  # Loop over every item in the dataset
  for i in range(len(data)):
    vec2=data[i]

    # Add the distance and the index
    distancelist.append((cosine(vec1,vec2),i))

  # Sort by distance
  distancelist.sort()
  return distancelist

\end{mylisting}

I implemented the cosine function manually instead of using one of the many math pacakges because it was a short method and I'm running off a Virtual Machine after my laptop died. I am not happy with this method, this is where I thought I could get rid of the title matching itself as a neighbor. If the distance on the bottom equaled zero, I believe th entries would all be the same. I assumed from this that the blogs would be the same entry, so I set the distance high. The function is shown below:
\begin{mylisting}[hbox,enhanced,drop shadow]{Linking by index}

def cosine(v1,v2):
    sumTop=0.0
    for i in range(len(v1)):
        sumTop += v1[i]*v2[i]
    sumBottom1 = 0.0
    sumBottom2 = 0.0
    for i in range(len(v1)):
        sumBottom1 += v1[i] ** 2
        sumBottom2 += v2[i] ** 2
    if(sumBottom1 * sumBottom2 == 0):
        return 100
    else:
        return (sumTop)/((sumBottom1 * sumBottom2)**0.5)
\end{mylisting}

	Lastly, the code can be run with the run.py script, shown below. I went through several iterations of this trying to figure out what worked best. In the end I went for readability.
	
	\begin{mylisting}[hbox,enhanced,drop shadow]{Linking by index}
import clusters
import numpredict
def findNeigh(i, data, k):
    testing = data[i]
    neighbors = numpredict.knnestimate(data, testing, k)
    for i in neighbors:
        print(blogs[i[1]])


blogs, text, data = clusters.readfile("blogdata1.txt")
#print(blogs.index("F-Measure"))
#findNeigh(blogs.index("F-Measure"), data, 5)
for name in "F-Measure", "Web Science and Digital Libraries Research Group":
    for k in 1, 2, 5, 10, 20:
        print("Running", name, "against k =", k)
        findNeigh(blogs.index(name), data, k)
        print("\n\n")

\end{mylisting}
\section{Problems with the code}

The code for every example I found, found itself as a neighbor. I could not figure out why it was happening, I understand why in theory, but I could not figure out a way around it. Beyond this, both examples found the same nearest neighbor, other than themselves. I believe this is because the data is bad, not because the code is. If I had to do it again, I'd grab a new dataset.

\section{Example Output}

I've left the entire output in the file output.txt. However, I clipped F-Measure's neighbors with k=5 below.

Running F-Measure against k = 5

The World's First Internet Baby

CardrossManiac2

SPIN IT RECORDS Moncton 467A Main Street Moncton NB CANADA

F-Measure

Abu Everyday

\end{document}